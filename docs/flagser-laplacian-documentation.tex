\documentclass{article}
\usepackage{graphicx} % Required for inserting images

\title{flagser-laplacian Documentation}
\author{Ben Jones}
\date{September 2023}

\begin{document}

\maketitle

\section{Introduction}
flagser-laplacian computes the spectra of the Persistent Directed Flag Laplacian, defined on direfect flag (clique) complexes. It relies on substantial portions of the software flagser, by Daniel L\"utgehetmann \cite{lutgehetmannComputingPersistentHomology2020}, to build the directed flag complexes and (co)boundary matrices. 

This software takes as input a (filtered) directed graph and outputs the computed spectra. It does not produce a directed graph from point cloud data.

\section{Requirements}
flagser-laplacian requires a C++17 compiler, CMake, and a MATLAB installation.

This software uses the MATLAB Engine API for C++, may be difficult to use in a Windows environment.

This software also uses the C++ library Eigen for some matrix computations. A full copy of Eigen is included in include/Eigen. 

\section{Installation and building}
You can install flagser-laplacian by first cloning the repository:

\begin{verbatim}
    git clone https://github.com/bdjones13/flagser-laplacian
\end{verbatim} 

\vspace{1em}
\textbf{Important:} the MATLAB directory in CMakeLists.txt (lines 21 and possibly lines 27 and 33) must agree with your MATLAB installation.
\vspace{1em}

Now you can compile the source code from within the repository: 

\begin{verbatim}
    > mkdir build
    > cd build
    > cmake ..
    > make
\end{verbatim}

This will produce the executable file \verb|flagser-laplacian|. If you want to use flagser-laplacian from a different directory, you may want to add the directory containing the executable file to your path.

\section{Usage}

\subsection{Input data}
The input data file for flagser-laplacian must be in the flagser format:

\begin{verbatim}
    dim 0:
    filtration_vertex_0 filtration_vertex_1 ... filtraition_vertex_n
    dim 1:
    first_vertex_id_of_edge_0 second_vertex_id_of_edge_0 filtration_edge_0
    first_vertex_id_of_edge_1 second_vertex_id_of_edge_1 filtration_edge_1
    ...
    first_vertex_id_of_edge_m second_vertex_id_of_edge_m filtration_edge_m
\end{verbatim}

Note that the filtration values are not optional. If you do not want to use a filtration, use a filtration value of $0$.

\subsection{Running flagser-laplacian}
\begin{verbatim}
    ./flagser-laplacian [options] datafile.flag
\end{verbatim}

where datafile.flag is a file in the flagser format described above.

Here is a description of the possible options:

\begin{itemize}
    \item[\textbf{--out-prefix prefix}:] is an optional parameter that will add \verb|prefix| to the beginning of the output file names. This is useful when making multiple calls to flagser-laplacian.
    \item[\textbf{--max-dim dim}:] is an optional parameter that will limit the dimension of the spectra to be computed.
\end{itemize}

Here are some examples of ways to call flagser-laplacian

\begin{itemize}
    \item \verb|./flagser-laplacian a.flag|
    \item \verb|./flagser-laplacian test/a.flag|
    \item \verb|./flagser-laplacian --out-prefix myprefix test/a.flag|
    \item \verb|./flagser-laplacian --max-dim 2 test/a.flag|
    \item \verb|./flagser-laplacian --max-dim 2 --out-prefix myprefix2 test/a.flag|
\end{itemize}

Other options that flagser implemented are not currently available in flagser-laplacian.


\subsection{Output}

The number of output files will depend on the directed graph you input and if you specified a maximum dimension. There will be an output \verb|prefix_spectra_i.txt| for each dimension $i$ of the directed flag complex, up to the maximum dimension you specify. There will also be an output \verb|prefix_spectra_summary.txt|. 

The format for a file \verb|prefix_spectra_i.txt| will be of the form:

\begin{verbatim}
    >
    > 
    > 1
    > 0 0 2
    > 0 0 0 0.1 1.0
\end{verbatim}

Each line represents the spectra of  $\Delta^{a,b}_i$, where $a$ is the filtration value corresponding to the row number, and $b$ is the next filtration value. For example, if the filtration is $0$, $1.5$, $2$, $3$, the 2nd line of the file will report the spectra for $\Delta^{1.5,2}_i$. \textbf{Note:} the final line of each \verb|spectra_i.txt| file is equal to the spectra of $\Delta^b_i$, where $b$ is the final filtration value, e.g. $3$ in the above example. This final step is achieved by inserting an artificial final filtration value of $b+1$, and since the directed flag complexes at $b$ and $b+1$ are equal, we will have $\Delta^b_i = \Delta^{b,b+1}_i$. Observe that the first lines may be blank if there are no directed $i$-cliques at that filtration step. The spectra are printed in non-decreasing order. 

The other output file, \verb|prefix_spectra_summary.txt|, is in tab separated value format. The headers are $i$ (filtration index starting from $0$), \verb|filtration| (the real number filtration values), \verb|betti_0|, \verb|betti_1| ... \verb|bett_N| (persistent Betti numbers up to the top dimension of the complex or the specified maximum $N$), \verb|lambda_0|, \verb|lambda_1| ... \verb|lambda_N| (least nonzero persistent eigenvalues). If there are no nonzero eigenvalues, at that filtration level, \verb|lambda_i| is taken to be $0$. It might look like this:

\begin{verbatim}
    i   filtration  betti_0 lambda_0
    0   0   1   0
    1   0.1 2   0.23
\end{verbatim}

Note that each tab separator is a tab. Depending on the software you use to view this file, it may not appear as aligned columns.
%%  The bibliography
\bibliographystyle{abbrv}
\bibliography{flagser-laplacian}
\end{document}
